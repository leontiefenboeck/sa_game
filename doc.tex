% S&A Latex Template for Assignment Docs
% Reinhold Preiner, 2024
%-------------------------------------------------------------------

\documentclass{article}
\usepackage[utf8]{inputenc}
\usepackage{hyperref}
\usepackage{cite}
\usepackage[english]{babel}
\usepackage{hyperref}
\usepackage{subfig}
\usepackage{graphicx}
\usepackage{float}

\usepackage{geometry}
\geometry{margin=1in}

\hypersetup{
    colorlinks=true,
    linkcolor=blue,
    filecolor=magenta,      
    urlcolor=cyan,
    citecolor = black,
}

\setlength{\parindent}{0pt}
\setlength{\parskip}{0.5em}

\usepackage{comment}

\title{	
	\large Simulation \& Animation - SS 2025\\
	\Huge{Binding of glass Mini golf 2}\\
	\huge{Can you hit it?}
}
\author{\parbox{\textwidth}{\centering
	Florian Winston, 11727495, winston@student.tugraz.at\\%
	Leon Tiefenboeck, 11919874, tiefenboeck@student.tugraz.at\\%
}}
\date{\today}


\begin{document}

\maketitle

\section{Game Documentation}

So far we have implemented two levels, where each showcases 
a different technique, however for the levels to be really complete and make sense to play the 
second technique is needed (per level). 

The game is started by opening \texttt{main.html} which opens a 
start screen where one can choose between the two levels. 
The main concept is the same in both levels: 
In the bottom of the game area there is a blue ball that can be moved 
by clicking it, dragging somewhere and then releasing the mouse button. 
The ball will move in the opposite direction of where was dragged and the momentum of the 
ball depends on how far back was dragged, much like a spring. 
Somewhere else in the game area there is a green circle. This is the hole where 
the ball should go into. If the player manages to get to the ball into the hole the level is completed 
and they are put back into the main menu. 
On the left side of the screen there is the control panel where level specific techs 
and the overall animation update rate can be adjusted. Here one can also toggle the needed visualizations 
for different techs. Now below we will outline the different levels and their techs. 

\subsection{Level 1}

This level so far features the Path Interpolation tech. However this will really only make sense 
once Rigid Bodies are also implemented, since the moving objects should act as obstacles where the 
ball bounces off of. For this reason we only added two splines to showcase, one that forms a wave and the other
just a straight line. As for now the movement only loops but we plan to add the option to 
make them move back and forth. 

\subsection{Level 2}

\section{Technical Specifications}

We implemented our game very modular way. We have one main game class that controls the game logic 
and everything that is the same in each level. This class also includes one main loop that 
updates the position of objects and renders them. We can very easily add other objects from other parts in the 
code to this loop. We can adjust the animation update rate here 
by simply adding a delay to when the next iteration of this loop is called. 
We achieve the framerate independence also here, by calculating how long a frame took to produce 
and then adjust all updates by this delta. 

Next we have an individual class for each of the techs. Here everything should 
also be confined and other parts of the game need only interact with the constructor
and the \texttt{update} and \texttt{render} methods.

Finally we have a file for each level that interacts with the main game 
class and the different techs. Here we set up
the objects and their techs in the respective level and the controls that are level specific (visualization, speed adjustment, etc.).
If everything is set up we add the \texttt{update} and \texttt{render} methods of the objects (that come from the techs) 
to the main game loop and the start the game by calling the loop.








\end{document}

